Problem statement
\newline
Kalafong Provincial Tertiary Hospital is yet not computerized, meaning Clinical records of patients are kept in a paper based platform (which is clinical files), and this way of doing things it makes things extremely difficult, for instance to get access to accurate data with regards to the patients information, even though the data is accurate but going through the process of gaining access to the information can be very time consuming and it affect research because information cannot be accessed anytime, anywhere and by any means of technology we have in hand.

\newline
\textbf{Overview of current system}
\begin{itemize}
	\item Clinical records of patients are recorded in an A4 paper by the people on duty.
	\item The information and its accuracy at times may be hard to verify and referenced to the person who recorded it because the identity of the person is not automatically stipulated to that paper.
	\item It is easy to lose the clinical records of the patients all at once in case of fire, natural disasters and human error factors.
\end{itemize}

\newline \\

\textbf{Proposed Solution}
\newline \\
\begin{itemize}
	\item First, all the users of the system they will be registered and assigned different access level based on their position (e.g. Medical interns, senior medical students, Research assistants etc.).
	\item Allow users to record information using different online web based platforms (smartphones, tablets, laptops and desktop computers) and automatically referenced the username of the person entered the record.
	\item Restrict users to see the information except those authorized to do so.
	\item Integrate this system will the already existing department Microsoft Access database.
	\item Link this system with other online systems (e.g. National Health Laboratory System) that may need its services and vice versa.
	\item Allow users to search patients’ records using certain parameters (In a usable and faster interface).
\end{itemize}

\newline \\ \\
This will make the application interacting with our data system able to adhere to the following usability goals:
\\
\begin{enumerate}
    \item Effectiveness - make the product good at what it’s supposed to do.
    \item Efficiency - help to increase productivity.
    \item Utility - provide the functionality that the users
    want/need.
    \item Learnability - Make it easy for user to learn and use the product.
    \item Intuitive - Make it easy to use and understand the application.
    
\end{enumerate}
